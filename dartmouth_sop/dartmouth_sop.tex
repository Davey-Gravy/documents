\documentclass{letter}
\usepackage[a4paper, left=1.85cm, right=1.85cm, top=1cm, bottom=1cm]{geometry}
\usepackage[version=4]{mhchem}

\signature{Davis Cole}
\address{8 Reservoir Road\\ Lebanon, NH 03766 \\ davisrcole@gmail.com}
\begin{document}

\begin{letter}{Graduate Admissions \\ Thayer School of Engineering \\ Dartmouth College \\ 15 Thayer Drive \\ Hanover, NH 03755}
    \opening{Dear Thayer School of Engineering Graduate Admissions,}

    My name is Davis Cole and I am a R\&D Verification Engineer
    at Ansys Inc. in Lebanon. Prior to starting this position, 
    I worked at DEKA R\&D in Manchester as a test engineer, 
    aiding in the development of state-of-the-art medical devices. 
    Despite these titles, I have done work for various engineering
    disciplines to progress project goals. I am confident that I 
    would benefit from graduate studies at Dartmouth by advancing 
    my knowledge in mechanical engineering subjects including 
    thermofluids and applied mathematics, and further developing 
    other skills in software engineering such as high-performance 
    computing. The Fluent computational fluid dynamics (CFD) code 
    I work with on a daily basis was developed in part by Dartmouth 
    and other affiliates, and I cannot think of a better place to 
    advance my education.

    I was introduced to CFD in my junior year at the University of 
    New Hampshire (UNH) through an introductory course. After learning 
    about various discretization schemes, solution methods, and turbulence 
    modeling, I was introduced to OpenFOAM, creating toy simulations that 
    demonstrated convection, diffusion, and other physical phenomena. 
    Following this course, I was able to apply what I learned at DEKA R\&D 
    as an intern, primarily focused on modeling flow through deformed IV tubes. 
    This employed a form of na\"ive fluid-structure interaction, where a 
    separate finite-element analysis package modeled hyperelastic deformation 
    of the IV tubes, which were exported to OpenFOAM for CFD analysis. Having 
    seen the utility of engineering simulation first-hand in performing root 
    cause analysis and iterative design, I became incredibly interested in 
    this field.

    I then sought out a capstone project that involved CFD, and I became the 
    lead for a research project investigating the trajectory of aerosolized 
    COVID-19 particles in the built environment, with a focus on classrooms 
    and methods to mitigate lateral transfer to prevent infection. Under the 
    assumption that aerosols behave as a passive scalar in a flow field, we 
    designed experiments using \ce{CO_{2}} as a passive scalar. This project 
    coincided with the pandemic, and access to classrooms was limited, 
    demonstrating an opportunity to develop a digital twin of the classrooms. 
    This resulted in my first experience with Ansys Workbench and Fluent. 
    The model I developed was critical for rapid prototyping of transfer 
    solutions, reducing time spent setting up experiments and lowering material 
    expenses.

    In parallel with my capstone, I took an elective in high-performance computing. 
    This taught me scientific computing with C++, version control with Git, and 
    parallelization with OpenMP and MPI. Given that software engineering was not 
    emphasized in UNH's mechanical engineering curriculum, there was a learning 
    curve to overcome, but as my final project I chose to simulate the 2D 
    lid-driven cavity problem using the SIMPLE algorithm to connect the course to CFD.

    Prior to graduation, I was offered a position at DEKA R\&D as a mechanical 
    engineer to continue working on the same project as my internship. However, 
    I had spent two weeks in Iceland for a brief sustainability study abroad 
    program, and I was inspired by their pervasive use of renewable energy. 
    Having attended lectures at Reykjavik University during this program, I 
    decided to enroll and pursue a Master's degree in Sustainable Energy 
    Engineering, in hopes of contributing to the clean energy transition 
    stateside. While I had no issues academically, I quickly learned that I 
    require more than four hours of sunlight per day, resulting in my return 
    to the US after one semester. Regardless, I found my experience to be 
    incredibly valuable. I continue to have a strong interest in sustainability, 
    and I hope to apply engineering analysis to renewable energy systems in the future.

    Having been born and raised in Lebanon, Dartmouth has been a constant presence 
    in my life, with my grandmother having served as the ski team's secretary, and 
    my father being an alumnus. Due to this, attending Dartmouth has been a bit of 
    a dream for me, and I would be ecstatic to continue this legacy for another generation.

    Thank you for your time and consideration, and I look forward to your response!

    \closing{Yours Faithfully,}

\end{letter}
\end{document}
